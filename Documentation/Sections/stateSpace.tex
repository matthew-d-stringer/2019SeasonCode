\documentclass[../main.tex]{subfiles}
% !TEX root = ../main.tex

\begin{document}
    \section{State Space Modeling and Control}

    \subsection{Motor Modeling}
    When doing analysis DC motors, we represent them mathematically as a resistor
    and generator in series.
    \begin{equation*}
        V = IR + \dot \theta_m k_v
    \end{equation*}
    For motors, it is assumed that the torque is proportional to the current.
    \begin{equation*}
        \tau_m = I k_t
    \end{equation*}
    With this knowledge we can find the torque comming from a motor
    \begin{equation*}
        \tau_m = -\frac{k_v k_t}{R} \dot \theta_m + \frac{k_t}{R} V
    \end{equation*}
    If we have a gearbox attached to our motor, we know that 
    \begin{align*}
        \dot \theta = G \dot \theta_m \\
        \tau = \frac{\tau_m}{G}
    \end{align*}
    where $G$ is the gear ratio. 
    Replacing this into the original formula gives us 
    \begin{equation*}
        \tau = -\frac{k_v k_t}{RG^2} \dot \theta + \frac{k_t}{RG} V
    \end{equation*}

    \subsection{Mechanism Modeling}
    The model above allows you to model the many possible mechanisms on your robot as long as you 
    have a basic understanding of high school physics. In order to make sure you have a useful 
    model, check that you have a differential equation. Here we are going to go into detail on modeling 
    specific models. These models will then be put into State Space Representation in order to conduct 
    further analysis.

    \subsubsection{Motor with Mass Model}
    In this example, we are going to model a motor that has a mass attached to the output shaft. This 
    motor will be attached to a gearbox that has the gear ratio $G$. In order to model this system,
    we will attempt to find the acceration of the mass. We can do this by finding the net torque of the system
    as follows.
    \begin{equation*}
        \sum \tau = L \ddot \theta = -\frac{k_v k_t}{RG^2} \dot \theta + \frac{k_t}{RG} V
    \end{equation*}
    Where $L$ is your moment of inertia. Finally, your model will be
    \begin{equation*}
        \ddot \theta = -\frac{k_v k_t}{LRG^2} \dot \theta + \frac{k_t}{LRG} V
    \end{equation*}

    \subsubsection{Motor attached to Arm Model}

    \subsection{State Space Representation}

\end{document}